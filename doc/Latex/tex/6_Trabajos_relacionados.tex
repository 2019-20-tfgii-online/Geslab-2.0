\capitulo{6}{Trabajos relacionados}

Con respecto a trabajos relacionados que puedan existir en el mercado no hay mucho que mencionar, ya que actualmente no existe ninguna producto comercial que permita el funcionamiento particular que solicita la facultad de ciencias. es importante destacar que la aplicación no gestiona las ventas de productos, únicamente es una herramienta de consulta de existencias, para el control de facturas y herramientas se lleva a través de otra aplicación de la Universidad.

Por ello la única herramienta que hay que mencionar es la primera versión de la misma:

\subsection{Geslab 1.0}
Lo primero que hay que mencionar, es la primera versión de la misma aplicación, creada por Alvaro Luis de Miguel en 1999 como su trabajo de fin de grado \cite{GeslabV1}. 

Aunque es una aplicación desarrollada hace ya muchos años, muchos de los funcionamientos y de la manera de trabajar han sido trasladados a esta nueva versión, actualizando los conceptos necesarios para poder mejorar su funcionamiento.

