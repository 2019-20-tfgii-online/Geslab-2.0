\capitulo{1}{Introducción}

La gestión del inventario para una organización como la Facultad de Ciencias es una tarea de vital importancia. Una mala gestión de los materiales que tiene en sus laboratorios, puede suponer una pérdida recursos temporales/económicos para la Universidad.

Es importante que todos los departamentos empleen un sistema eficiente de gestión, y que todo el personal sepa utilizar este sistema de una manera correcta. Son los propios miembros de los laboratorios los encargados de registrar tanto las entradas como las salidas de material, por lo que el diseño de la aplicación que van a utilizar tiene peso importante. 

Si el usuario se tiene que enfrentar a una aplicación de gestión poco intuitiva y que requiera de un largo y complicado proceso de aprendizaje, probablemente termine frustrado y finalmente abandonará su uso.

El objetivo principal del proyecto consiste en una actualización del proyecto de Álvaro de Luis De Miguel \cite{GeslabV1}, cambiando la aplicación de escritorio que el desarrolló en 1999 por una aplicación web alojada en un servidor de la universidad.