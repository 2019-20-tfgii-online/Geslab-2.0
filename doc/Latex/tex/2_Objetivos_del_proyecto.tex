\capitulo{2}{Objetivos del proyecto}

El objetivo principal del proyecto es la renovación de la aplicación ya existente en una aplicación web así como la importación de todos los datos que posee el sistema actual al nuevo. También se han de añadir distintos requisitos que con el uso de la aplicación, los usuarios han marcado como necesarios. Entre ellos se puede destacar un sistema que permita almacenar las medidas de seguridad necesarias para la conservación de los productos, así como un sistema capaz de unificar todos los posibles nombres que puede tener un producto, y evitar así problemas de elementos duplicados.

Antes de analizar en profundidad los nuevos requisitos, es importante explicar el funcionamiento actual de la aplicación.


\section{Funcionamiento actual de la aplicación}

Geslab 1.0 permite a un usuario creado por el administrador de la base de datos logearse y acceder a la información que se encuentra en la misma.

De esta manera el usuario puede consultar las existencias actuales de un producto, que se almacena en forma de \textbf{ficha de producto}. Cada ficha se corresponde con un producto guardado en una ubicación, con una calidad determinada, de una marca en concreto y proporcionado por un distribuidor.

De cada ficha puede haber tanto \textbf{entradas} como \textbf{salidas}, y de estas se guarda la fecha, su caducidad, el nº de lote, las unidades, su capacidad y si es residuo o no. Así, cada ficha puede tener varias entradas y salidas y su stock real se calcula con los datos de estas.

El sistema también almacena una serie de datos importantes para su funcionamiento, dejando estos a disposición del propio usuario para que pueda editarlos y añadir nuevos en caso necesario.   

Hay una tabla con todos los \textbf{productos} de los que puede disponer una ficha, junto a información relevante del producto (formula química, precauciones, etc).

De la misma manera se guardan las distintas \textbf{calidades} que puede tener el producto.

El sistema almacena tanto las \textbf{marcas} como los \textbf{proveedores}, añadiendo también información de contacto.

Por otro lado se almacenan los datos de \textbf{departamentos}, \textbf{áreas} y \textbf{centros}, pero a diferencia de los anteriores estos solo podrán ser editados por un usuario administrador de la base de datos.

Con todos estos datos, el usuario al logearse en la aplicación puede consultar las existencias, pudiendo ver las entradas y las salidas de cada ficha, y puede realizar una búsqueda filtrando en función de los muchos campos de los que disponen las tablas.


\section{Nuevo paradigma}

Antes de continuar con el análisis es importante mencionar como se ha realizado un cambio de paradigma en el uso de los objetos de la base de datos.

En la primera versión de Geslab se trata con una mayor importancia a las tablas de entradas y salidas, teniendo estas muchos campos para después poder calcular el stock de manera dinámica restando las salidas a las entradas. 

En la nueva versión esto cambia, pasando a tener mayor importancia la ficha de producto. Esta se lleva prácticamente todos los atributos que tenían las entradas y salidas además de un campo nuevo para almacenar el stock real.

De esta forma el stock siempre estará actualizado, y solamente se actualizará cada vez que haya una entrada o una salida. 


\section{Nuevas funcionalidades}

La actualización de la aplicación va a recibir nuevas funcionalidades, unas requeridas por el propio planteamiento de la actualización, otras requeridas por parte de los usuarios de la aplicación y otros funcionalidades no funcionales que también merece la pena destacar. 

Antes de nada hay que explicar que todos los requisitos mostrados a continuación son un resumen de los requisitos más destacados, y los que se diferencian de la versión anterior de la aplicación. La relación detallada de todos ellos se puede encontrar en los anexos.

\subsection{Requisitos del proyecto}

\begin{itemize}
\item El proyecto ha de mantener todos los requisitos actuales de la aplicación.

\item También se requiere un sistema que permita almacenar las medidas de seguridad necesarias para la conservación de los productos.

\end{itemize}

\subsection{Requisitos de los usuarios}

\begin{itemize}

\item Tienen que existir varios roles de usuario, \textbf{administrador de la aplicación}, que puede editar las tablas y su funcionamiento, \textbf{gestores de inventario}, que pueden gestionar las fichas de productos, y \textbf{usuarios}, que solo pueden ver la información sin conocer la ubicación de estos.  

\item Se han de añadir los siguientes campos a la tabla producto: pureza, peso molecular, formula desarrollada, Nº CAS, Nº EINECS, Nº EC, pictogramas de seguridad, indicadores de peligro, indicadores de prudencia y hoja de seguridad (PDF)

\item El sistema debe de permitir una búsqueda simplificada y una búsqueda avanzada. La búsqueda avanzada deberá de incluir los campos marca, fórmula, CAS, caducidad, fecha adquisición, nombre y localización. El sistema no deberá distinguir entre mayúsculas y minúsculas.

\item El sistema debe de contemplar los distintos sinónimos que poseen los productos, así como identificar el compuesto tanto por su nombre en español como en inglés.
  
\end{itemize}

\subsection{Requisitos no funcionales}

\begin{itemize}

\item La aplicación se desarrollará en lenguaje java con servlets y jsp y se realizará a través de Eclipse.

\item La base de datos se creará en MySql, y se conectará con Java a través del MySql Connector.

\item El diseño de la aplicación se realizará en HTML5 y se utilizará bootstrap 4 

\end{itemize}

