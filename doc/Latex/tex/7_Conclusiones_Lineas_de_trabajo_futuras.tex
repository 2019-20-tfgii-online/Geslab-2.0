\capitulo{7}{Conclusiones y Líneas de trabajo futuras}

\section{Conclusiones}

Tras meses de trabajo y con una cuarentena de por medio, finalmente se ha conseguido obtener un producto funcional, que cumple los requisitos marcados de una manera satisfactoria. 

Espero que este trabajo pueda servir de una ayuda real a todos los miembros de los laboratorios, facilitando un poco su día a día.

A nivel personal, el desarrollo de Geslab 2.0 me ha unido de nuevo a muchos sectores de la ingeniería informática que tenía olvidados. Bases de datos, programación en Java, diseño web, todos son aspectos que he refrescado y aprendido en más profundidad, y que me hacen replantearme mis oportunidades laborales de aquí en adelante.

Pero no solo la parte técnica, el poder llevar a cabo un proyecto desde cero, con todo lo que ello implica, me ha acercado a lo que me voy a poder encontrar en cualquier empresa que desarrolle productos de software.

\section{Líneas de trabajo futuras}

\subsection{Herramienta importar y exportar datos} 
Actualmente, los datos de productos no solo se encuentran en la base de datos de la aplicación, los usuarios disponen de sus propios documentos en los que han ido guardando la información. Por ello, una herramienta que pueda facilitar la tarea tanto de importar como de exportar datos sería muy útil en la realidad.

\subsection{Herramienta exportar etiquetas}
Los productos del laboratorio en ocasiones requieren de ser acompañados por una etiqueta de seguridad. Toda esta información esta almacenada en la aplicación, por lo que una herramienta para poder imprimir las etiquetas con un mismo formato, y pudiendo editar los valores sería de gran valor. 

\subsection{Versión móvil}
Hoy día la tecnología está cada vez más enfocada a los dispositivos móviles, por lo que hay que pensar en una forma de adecuarlo a estos tipos de dispositivos. El aspecto gráfico de la aplicación esta creado con bootstrap, lo cual permite crear un diseño responsive. Actualmente solo esta optimizado para dispositivos de escritorio, pero queda la opción de actualizarlo para que sea responsive, o por otro lado, también se puede implementar una API para convertirlo en una aplicación móvil. 


\section{Agradecimientos}

Por último, me gustaría dar las gracias a todos los que me han ayudado en el desarrollo de Geslab 2.0, a los miembros del laboratorio, a Gonzalo, decano de la facultad, a Pedro mi tutor, por lucharlo cada semana junto a mí, a mi familia y a mi pareja por aguantar mi convivencia en unos meses de tanto estrés. 

Gracias.