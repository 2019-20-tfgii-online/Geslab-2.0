\capitulo{3}{Conceptos teóricos}

En este apartado se hará hincapié en distintos conceptos teóricos que son necesarios para la comprensión del proyecto. Es importante analizar los cambios de tecnología que se van a realizar, para poder comprender correctamente su comportamiento. 

\section{Conceptos químicos}

Al tratar el proyecto sobre una rama como la química, hay muchos conceptos que pueden no resultar conocidos para cualquier persona, por lo que a continuación se definirán todos los campos nuevos que se van a añadir en el sistema.

\begin{itemize}

\item \textbf{Pureza:} grado de descontaminación de un reactivo o producto. Se expresa como porcentaje.

\item \textbf{Peso molecular:} suma de todas las masas atómicas de los átomos presentes en una molécula. Se expresa en gramos por cada mol de sustancia (g/mol).

\item \textbf{Formula desarrollada:} representación gráfica de la estructura molecular. Enlace a imagen del fabricante.

\item \textbf{Nº CAS:} identificación numérica única para compuestos químicos, polímeros, secuencias biológicas, preparados y aleaciones. Se expresa como tres grupos de números separados por guiones (XXXXXXX-XX-X).

\item \textbf{EC/EINECS/ELINCS Index Number:} número de registro dado a cada sustancia química comercialmente disponible en la unión europea. Se expresa como un sistema de números de siete dígitos (XXX-XXX-X).

\item \textbf{Enzyme Commission Number:} esquema de clasificación numérica para las enzymas con base en las reacciones químicas. Se expresa como dos letras \textit{EC} seguidas de cuatro números separados por puntos (X.X.X.X).

\item \textbf{Pictogramas de seguridad:} imágenes adosadas a etiquetas que incluyen un símbolo de advertencia y colores específicos con el fin de transmitir información sobre el daño que una determinada sustancia o mezcla puede provocar a la salud o al medio ambiente. Cada pictograma esta definido, además de por la propia imagen, por un número de referencia y una descripción de la imagen que lo representa. La referencia se expresa como las letras \textit{GHS} seguidas de un número de dos cifras (GHSXX) \cite{Seguridad}.

\item \textbf{Indicadores de peligro, para abreviaturas de frases H:} son frases que, asignadas a una clase o categoría de peligro, describen la naturaleza de los peligros de una sustancia o mezcla peligrosas, incluyendo cuando proceda, el grado de peligro. Se pueden expresar de varias formas, generalmente como una H seguida de un número de tres cifras (HXXX), en otras ocasiones también pueden llevar las letras \textit{EU} delante para informar de indicaciones suplementarias (EUHXXX) y por último se le pueden añadir letras de tres dígitos al final (HXXXYYY) \cite{Seguridad}.


\item \textbf{Indicadores de prudencia, para abreviaturas de frases P:} son frases que describen las medidas recomendadas para minimizar o evitar los efectos adversos causados por la exposición a una sustancia o mezcla peligrosa durante su uso o eliminación. Se expresa como una P seguida de un número generalmente de tres cifras (PXXX), aunque puede mostrase también como suma de dos (PXXX+PXXX)  \cite{Seguridad}.

\end{itemize}

\section{MySQL}

MySQL es un sistema de gestión de base de datos relacionales de código abierto con un modelo cliente servidor. Actualmente está considerada como la base de datos open source más popular del mundo sobre todo para entornos de desarrollo web.

La aplicación actual utiliza como sistema gestor de base de datos el SQL Server 6.5 de Microsoft en inglés, mientras que la  nueva actualización se desarrollará en MySQL. Geslab 1.0 esta desarrollada como una aplicación de escritorio de windows, por lo cual es razonable que se eligiera SQL Server.

Con el salto a aplicación web el cambio a MySQL viene de la mano, ya que MySQL es mucho más sencillo de emparejar con cualquier otro idioma, como en nuestro caso es Java.


\section{JSP}

JavaServer Pages (JSP) es una tecnología que ayuda a los desarrolladores de software a crear páginas web dinámicas basadas en HTML y XML. JSP es similar a PHP, pero usa el lenguaje de programación Java \cite{wiki:JSP}

El motor de las páginas JSP se base en servlets, que son programas destinados a ejecutarse en el lado del servidor de modo que puede ampliar las capacidades del mismo.

Por todo ello, la utilización de JSP y Servlets nos permitirá realizar de una manera muy cómoda todo el diseño de las páginas HTML que utilizará la aplicación, evitándonos el tener que escribir infinitas sentencias println. 

Además esto nos ayuda a diferenciar bien la tarea del diseño gráfico de la aplicación por un lado,y por otro todo el comportamiento de programación.

\section{Maven} 

Maven es una herramienta de software para la gestión y construcción de proyectos Java con un modelo de construcción basado en XML. Maven utiliza un POM para describir el proyecto software a construir, sus dependencias de otros módulos y componentes externos, y el orden de construcción de los elementos. 

Además, el motor incluido en su núcleo puede dinámicamente descargar plugins de un repositorio, el mismo repositorio que provee acceso a muchas versiones de diferentes proyectos Open Source en Java, de Apache y otras organizaciones y desarrolladores. \cite{wiki:maven}

Maven nos ayuda mucho a la hora de crear un proyecto web gracias a su posible implementación con Eclipse. La configuración de las dependencias se realiza de manera muy sencilla, desde la configuración de los servlets, a la configuración de tomcat en el proyecto.
